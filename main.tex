\documentclass{amia}
\usepackage{graphicx}
\usepackage[labelfont=bf]{caption}
\usepackage[superscript,nomove]{cite}
\usepackage{color}
\usepackage{wrapfig}
\usepackage{lipsum}

\begin{document}

\title{Semantics from the Questionnaire Up}

\author{Joseph R. Utecht, B.A.$^{1}$, Firstname B. Lastname, Degrees$^{2}$}

\institutes{
    $^1$University of Arkansas for Medical Science, Little Rock, AR, USA; $^2$Institution, City, State, Country (if applicable)\\
}

\maketitle

\noindent{\bf Abstract}

\textit{Abstract text goes here, justified and in italics.  The abstract would normally be one paragraph long.  See Table 1. for appropriate abstract length by submission type.}

\section*{Background}
\lipsum[1-3]

\section*{Another Way Forward}
\lipsum[4-9]

\section*{How to Represent Questions}
\lipsum[10-15]

\section*{Advantages Over Previous Methods}
\lipsum[16-19]

\section*{Problems and Limitations}
\lipsum[20-21]

\section*{Our Implementations}
\lipsum[21-30]

\section*{Conclusion}
\lipsum[30-33]

\makeatletter
\renewcommand{\@biblabel}[1]{\hfill #1.}
\makeatother

\bibliographystyle{unsrt}
\begin{thebibliography}{1}
\setlength\itemsep{-0.1em}

\bibitem{ref1}
Pryor TA, Gardner RM, Clayton RD, Warner HR. The HELP system. J Med Sys. 1983;7:87-101.
\bibitem{ref2}
Gardner RM, Golubjatnikov OK, Laub RM, Jacobson JT, Evans RS. Computer-critiqued blood ordering using the HELP system. Comput Biomed Res 1990;23:514-28.



\end{thebibliography}

\end{document}

